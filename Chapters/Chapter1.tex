% Chapter 1

\chapter{Introduction} % Main chapter title

\label{Chapter1} % For referencing the chapter elsewhere, use \ref{Chapter1} 

%----------------------------------------------------------------------------------------

% Define some commands to keep the formatting separated from the content 
\newcommand{\keyword}[1]{\textbf{#1}}
\newcommand{\tabhead}[1]{\textbf{#1}}
\newcommand{\code}[1]{\texttt{#1}}
\newcommand{\file}[1]{\texttt{\bfseries#1}}
\newcommand{\option}[1]{\texttt{\itshape#1}}

%----------------------------------------------------------------------------------------

\section{Motivation}
After completing my training as a software developer, I grew disenchanted with working in the IT sector, where creative freedom is pretty much limited to coming up with names for variables and functions … and there are even guidelines for those too! In the field of new media and in the Time-based and Interactive Media bachelor's program at Linz Art University, the programming skills I had acquired could be used for creative purposes and even enhanced. Now, I'm adding greater depth to my knowledge in the Interface Cultures master's program offered at Linz Art University.

%----------------------------------------------------------------------------------------

\section{Artistic Background}
In light of the fact that my previous interactive works had been detailed and comprehensive encounters with the body as interface and the technology required to make this happen, the next logical step was to link up this technology with a corporeal art form such as contemporary dance. And fortunately, this was the exact point in time that Linz Art University and the Ars Electronica Center were arranging this joint venture, which is what enabled me to develop projects in an technologically advanced environment.

With the use of advanced tracking systems the stage can be transformed to an ubiquitous interface as whole. Through this the performers are given control. This open system gives them time and space for improvisation. Custom interaction modes focused on playfulness should trigger their creativity and provide a tool to use and tease their skills as performers. 

\section{Artistic Aims}
These environments only live when a body is moving within them; without movement they are silent and dark. Similarly, our performances itself cannot happen without the environment. The environment is the stage, as well as the instrument upon which the performer is playing.

Interactive technology or performance technology exists only in an integral relationship with the performer, the choreography, the sound sources, and so forth. I don't question its validity, I don't doubt its efficacy; I do constantly question its appropriateness and value with every piece I make.

Composer and choreographer must give up a huge amount of control. The dancer or performer must be given much more control. Tasks must typically be devised collaboratively, with the technology in place. A lot of time must be spent simply learning a new performance environment, usually at the same time that environment is being created. Everybody involved must be willing to experiment an be willing to cooperate in the creation of a performance environment that is also the piece itself.

Dance technologies demand an experimental approach to building performance environments, which create an entirely new collaboration between the \textbf{coders} and the performers. The piece is not finished until all the elements are in place and \textbf{in code}. The making of a performance environment has therefore a significant impact on the creative process.
